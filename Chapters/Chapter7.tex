
\chapter{Conclusions} % Main chapter title

\label{Chapter7} % Change X to a consecutive number; for referencing this chapter elsewhere, use \ref{ChapterX}

%----------------------------------------------------------------------------------------
%	SECTION 1
%----------------------------------------------------------------------------------------

\section{Summary and Findings}

This dissertation seeks to improve the performance of a downwind turbine using information obtained from an upwind turbines. The preceding chapters investigate how an upwind turbine can be used as a sensor, then develop and test two possible feed forward control schemes. The feasibility of each control schemes is investigated using a variety of test cases and a variety of wind turbine simulation tools.

Chapter \ref{Chapter2} investigates using a wind turbine as a sensor. It discusses the sensors that are present on a typical wind turbine and how sensor data can be used to gather useful information for feed forward control. Much of the chapter focuses on estimating wind speed. Anemometer wind speed measurements and wind speed estimates based on an inverted dynamic model of the turbine are examined. The model based estimator is found to have better performance, giving more accurate estimates of the average wind speed across the turbine rotor. The discrepancy between model based wind speed estimates and rotor average wind speed is approximately 3\% over most of the NREL 5-MW turbine's operating range. The effect of turbine yaw misalignment on model based wind speed estimation is also discussed. Yaw misalignment introduces a significant amount of error in wind speed estimates. However, the error can be compensated for if the amount of misalignment is known. Convection velocity is also discussed, as well as how to estimate it from wind speed estimates at a single location. Longer averaging times are found to produce more accurate convection velocity estimates, but there are practical limits on how long the averaging time can be. Finally, Chapter \ref{Chapter2} discusses other discusses other turbine sensors, such as wind vanes or rotor speed sensors, and how they could be useful for feed forward control.

In Chapter \ref{Chapter3}, a feed forward optimum pitch control scheme is introduced and investigated. In this control scheme an upwind is used as a wind speed sensor. Wind speed measurements are passed from the upwind turbine to a feed forward controller on a downwind turbine. The downwind turbine uses the knowledge of incoming wind speed to proactively adjust turbine blade pitch, potentially improving energy capture and reducing damage to the turbine. The control scheme is investigated through three rounds of simulation using the NREL FAST turbine simulation tool. The initial round of simulations uses a simplified idealistic model of the two turbine system, while the subsequent rounds systematically incorporate complicating factors that a real world system would encounter.  

In the first two rounds of simulations, feed forward control significantly improves turbine performance with respect to both extreme operating gusts and turbulent wind. However, the third round of simulations shows that the feed forward optimal pitch control system is very sensitive to timing errors. If the timing of the data is off by more than a few seconds the feed forward control system can cause a turbine to perform more poorly than a turbine without feed forward control. The system's high sensitivity to timing errors, and therefore the systems high sensitivity to errors in the estimation of the convection velocity, ultimately make this control scheme impractical in real world applications.

Chapter \ref{Chapter4} investigates the benefits and feasibility of derating a downwind turbine based on a feed forward signal from an upwind turbine. Several derating strategies from literature are investigated to determine which is most feasible for feed forward control. Derating a turbine by reducing the rated rotor speed is found to yield the largest reductions in structural loads and rotor overspeeds.The effects of derating on structural loading and rotor overspeed are examined. Reductions in rotor speed, power, blade root bending moments, and tower fore-aft bending moments are found to be roughly proportional to the amount of derating. The transition into and out of derated mode is also investigated. Though abruptly changing the rating of an NREL 5-MW turbine excited undesirable transient behavior, the undesirable behavior can be avoided by using a low pass filter. 

A feed forward selective derating scheme based on these findings is developed. That feed forward control system is then evaluated using a series of FAST simulations. The FAST simulations model a two turbine system in several wind conditions, including both IEC Extreme Operating Gusts and turbulent wind with large gusts. The simulation results are promising. The downwind turbine experiences significant reductions in tower fore-aft moments in all simulation cases, as well as reductions in blade root bending moments in some cases. More importantly, the downwind turbine does not experience any rotor overspeeds large enough to trigger an emergency shut down. The downwind turbine experiences reductions in energy capture, however those reductions are smaller than the reductions that would be caused by an emergency overspeed shut down.
  
Chapter \ref{Chapter5} is a verification and validation study of SOWFA (Simulator fOr Wind Farm Applications), a wind farm simulation tool. FAST must model a single turbine at time, but SOWFA can model a group of wind turbines as well at the atmospheric flow field surrounding those turbines. This gives SOWFA several advantages over FAST when modeling a two turbine system with feed forward control. However, SOWFA is much more computationally expensive than FAST and is less established in the wind turbine community. In Chapter \ref{Chapter5} the NREL 5-MW rotor is simulated using SOWFA, OVERFLOW2 and FAST. The results from these fundamentally different simulation tools are compared. Overall the results from OVERFLOW2 and SOWFA are found to have good agreement.  Power and thrust predictions are reasonably close for all test cases, as are the downstream momentum deficit predictions. The vortical structure of the wakes predicted by OVERFLOW2 and SOWFA are found to be very similar for the first 2\emph{R} downstream of the rotor though some differences are seen further downstream.

In addition, Chapter \ref{Chapter5} evaluates how near-rotor grid resolution and near-rotor grid length affect SOWFA simulation results. Lengthening the near rotor grid has no effect on wake behavior, though shortening the near rotor grid has a small effect. Increasing the near rotor grid resolution to 4 m slightly increases the power and thrust predicted by SOWFA. It also results in The loss of some detailed wake behavior, such as rotor tip vortices in the 8 m/s and 11 m/s test cases. However, the downstream momentum deficit predictions still have good agreement. Overall, the results documented in Chapter \ref{Chapter5} increase confidence in SOWFA and provide guidance for future SOWFA simulations.

In Chapter \ref{Chapter6}, Feed Forward Derating Control is simulated and evaluated using SOWFA. The first portion of the chapter documents background work required to simulate the control system in SOWFA. Feed forward derating control is implemented as a series of FORTRAN subroutines and inserted into the SOWFA source code. SOWFA$'$s actuator line model is tuned to produce good agreement with FAST. The trade offs involved in choosing appropriate computational domain sizes and grid resolutions are examined. In addition, a method for modeling large gusts in SOWFA is developed.

The second portion of the chapter documents SOWFA simulation results for two test cases. In the first test case one NREL 5-MW turbine is 20R downwind and 3R offset from another NREL 5-MW turbine. In the second test case one NREL 5-MW turbine is 20R directly downwind of and in the wake of another NREL 5-MW turbine. The turbines operate in a constant 12 m/s wind until an extreme coherent gust (ECG) increased wind speed to 25 m/s. Feed forward derating control significantly improves performance in both test cases. Rotor overspeeds and DELs ware significantly decreased while energy capture is dramatically increased by avoiding emergency overspeed shutdowns.  At a site with favorable conditions, feed forward derating control could provide a significant improvement in performance for a minimal financial investment.







%----------------------------------------------------------------------------------------
%	SECTION 2
%----------------------------------------------------------------------------------------

\section{Future Work}

The research presented in this dissertation demonstrates that feed forward derating control can provide significant performance improvements in the right conditions. However, there are several lines of future research that should be pursued. 

The first line of research is to continue modeling feed forward derating control in more sophisticated and more realistic scenarios. The simulations carried out in Chapter \ref{Chapter6} are more sophisticated and closer to real world operation than the simulations carried out in Chapter \ref{Chapter4}. However, there are still several real world factors that were not captured in these simulations. It would be beneficial  to study the complications that would be caused by larger numbers of turbines, changes in wind direction, complex terrain, and atmospheric turbulence.

In this dissertation feed forward derating control was modeled using a two turbine system where the wind direction was known. This made the turbine to turbine communication simple. It was clear which turbine was operating as a sensor and which turbine was using that sensor data for feed forward control. Turbine to turbine communication becomes more complicated when there are more than two turbines. In a wind plant each turbine would have many other turbines it could share information with. Determining which turbines should provide feed forward data to which other turbines is not a trivial task. Wind plants often have complex layouts and in many wint plants wind direction changes seasonally or over the course of each day. A plant level controller capable of directing turbine-to-turbine communication and adapting to changes in wind direction must be developed. 

The simulations in this dissertation did not examine the effects of complex terrain. Many wind plants, such as offshore wind plants or wind plants in the plains of the central united states, have simple flat terrain. Other wind plants, including many of the most productive wind plants in California, have complex hilly terrain. These hills can significantly affect both gusts of wind and turbine wakes. More research is needed to determine how complex wind affects feed forward derating control. 

Some of the simulations in Chapter \ref{chapter4} modeled atmospheric turbulence, but they did not model turbine wake effects or changes in wind speed fluctuations as they travel downstream. The simulations in Chapter \ref{Chapter6} modeled  turbine wake effects and changes in wind speed fluctuations as they travel downstream. Research has shown that turbulence has a significant effect on turbine wakes.\cite{troldborg2015, troldborg2010,troldborg2010b,madsen2010} More research is needed to determine how turbulence affects feed forward derating control.

The second line of research is to better quantify the benefits that can be achieved with feed forward derating control. One aspect of this research would be to work with a wind farm that has emergency overspeed shutdowns to determine how much energy is lost per year and how much feed forward derating control could increase energy capture. Another aspect of this research would be to better quantify structural damage caused by large gusts and how much repair costs could be reduced by feed forward derating control. This would require calculating the fatigue damage ($D_{fatigue}$) caused by large gusts. As discussed in Appendix \ref{AppendixC}, fatigue damage is an estimate of how much a component has been damaged as a fraction of it's life ($D_{fatigue}$ = 0 indicates a new part while $D_{fatigue}$ = 1 indicates component failure). Fatigue damage calculations could potentially be used to assign a monetary value to gust induced loads. Fatigue damage calculations are similar to Damage Equivalen Load calculations but can only be carried out if the ultimate load of the component is known. The NREL 5MW turbine is a fictional turbine and ultimate component loads do not exist, so they would need to be estimated. 